\def\matlab{MATLAB }
\def\simulink{SIMULINK }
\newcommand{\bbma} {\begin{bmatrix} }
\newcommand{\ebma} {\end{bmatrix}}
\newcommand{\bpma} {\begin{pmatrix} }
\newcommand{\epma} {\end{pmatrix}}


 \def\vec#1{\ensuremath{\mathchoice
                     {\mbox{\boldmath$\displaystyle\mathbf{#1}$}}
                     {\mbox{\boldmath$\textstyle\mathbf{#1}$}}
                     {\mbox{\boldmath$\scriptstyle\mathbf{#1}$}}
                     {\mbox{\boldmath$\scriptscriptstyle\mathbf{#1}$}}}}%

% tensor
\def\tens#1{\relax\ifmmode\mathsf{#1}\else\textsf{#1}\fi}

\def\norm#1{\|#1\|} 

\def\utilde{\undertilde}

%\def\kruskal#1{\left[\left[\; #1 \;\right]\right]}

\newcommand{\tucker}[2]{\left[ #1 \right] \cdot #2}
\newcommand{\kruskalone}[1]{\left[ #1 \right] \cdot \tens{I}}
\newcommand{\kruskal}[2]{\left[ #1 \right] \cdot #2 }
\newcommand{\krusk}[1]{\left[ #1 \right] }
\newcommand{\kruskaltwo}[2]{\left[ #1 \right] \cdot #2}

%\newcommand{\cprod}[3]{\left\langle\!\!\!\left\langle \, #1 \,|\,  #2  \right.\right\rangle_{#3}}

%\newcommand{\cprod}[3]{\left\langle #1 \,,\,  #2  \right\rangle_{#3}}

\newcommand{\cprod}[3]{\left\langle\,#1\,\left|\,#2\,\right.\right\rangle_{#3}}
\newcommand{\aprod}[3]{\left\langle\,#1\,\left|\,#2\,\right.\right\rangle^+_{#3}}
\newcommand{\bprod}[3]{\left\langle\,#1\,\left|\,#2\,\right.\right\rangle^\Box_{#3}}
\newcommand{\hprod}[3]{\left\langle\,#1\,\left|\,#2\,\right.\right\rangle^\boxplus_{#3}}

%\newcommand{\cprod}[3]{\left\langle\,#1\,|\cdot |\,#2\,\right\rangle_{#3}}


%\newcommand{\hprod}[3]{\left\langle\!\left\langle #1 \,|\, #2  \right.\right\rangle^*_{#3}}

\newcommand{\iprod}[2]{\left\langle #1 \,,\,  #2  \right\rangle}

\newcommand{\btens}[1]{ 
\underline{\tens{#1}}}

\newcommand{\htens}[1]{ 
{\tens{#1}^*}}

\newcommand{\ctens}[1]{ 
\utilde{\tens{#1}}}

%\newcommand{\atens}[1]{ 
%\underline{\underline{\tens{#1}}}}

\newcommand{\atens}[1]{ 
\tens{#1}}

\newcommand{\cvec}[1]{%\raisebox{1mm}{
\utilde}
}}

%commands for scalars 
\newcommand{\cscal}[1]{%\raisebox{1mm}{
\utilde}
}}
\newcommand{\bscal}[1]{%\raisebox{1mm}{
\underline}
}}

\newcommand{\amat}[1]{ 
\underline{\underline{\mat{#1}}}}

%\newcommand{\amati}[1]{ 
%\underline{\underline{\mat{#1}\hspace{-1pt}}}\hspace{1pt}}

\newcommand{\amati}[1]{ 
\mat{#1}}


\newcommand{\cmat}[1]{%\raisebox{1mm}{
\utilde}
}}


\newcommand{\btensi}[1]{ 
\underline{\tens{#1}\hspace{-1pt}}\hspace{1pt}}


\newcommand{\ctensi}[1]{ 
\utilde{\tens{#1}\hspace{-1pt}}\hspace{1pt}}

\newcommand{\bvec}[1]{ 
\underline{\vec{#1}}}



\newcommand{\bveci}[1]{
\underline{\vec{#1}\hspace{-1pt}}\hspace{1pt}}



\newcommand{\tmat}[1]{\mbox{mat} \left( 
\mathbf{#1} \right)}

\newcommand{\tvec}[1]{\mbox{vec} \left( 
\mathbf{#1} \right)}

\newcommand{\hvec}[1]{ \mathbf{#1}}


%\newcommand{\fone}[0]{\vec{\nu}_1 }
%\newcommand{\fzero}[0]{\vec{\nu}_0 }
%\newcommand{\fdcare}[0]{\vec{\nu}_{-}}

%\newcommand{\fone}[0]{\boxminus_1 }

\newcommand{\fone}[0]{\framebox[4mm]{1}}
\newcommand{\fzero}[0]{\framebox[4mm]{0}}
%\newcommand{\fdcare}[0]{\framebox[4mm]{\phantom{$0$}\hspace{-5pt}$-$}}
\newcommand{\fdcare}[0]{\framebox[4mm]{\phantom{$0$}}}



\newcommand{\mat}[1]{
\mathbf{#1}}

\newcommand{\bmati}[1]{ 
\underline{\mathbf{#1}\hspace{-1pt}}\hspace{1pt}}

\newcommand{\bmat}[1]{ 
\underline{\mathbf{#1}}}


\newcommand{\bighadamard}
{ \bigotimes\hspace{-3.26ex}\bigoplus}

\newcommand{\hadamard}
{\circledast}
%{ \otimes\hspace{-2.03ex}\oplus\;}

\newcommand{\vecop}[1]
{ \mbox{vec}\hspace{1mm}({#1}) }

\newcommand{\mycitation}[2]
{
\vspace{-25mm}
\hfill
\begin{minipage}[t]{0.45\textwidth}
\textit{#1}
\end{minipage}
\vspace{5mm}

\hfill \textrm{#2}

\vspace{15mm}
}


\newcommand{\tikzfig}[4]
{
\begin{figure}[h!]
\begin{center}
\scalebox{#3}{
\input{../pic/#2/#1.tikz}
%Figure
}
\end{center}
\caption{#4}
\label{fig:#2--#1}
\end{figure}
}


\newcommand{\myfig}[4]
{
\begin{figure}[h!]
\begin{center}
\ifpdf
	\includegraphics[scale = #3]{../pic/#2/#1.pdf}
\else
	\includegraphics[scale = #3]{../pic/#2/#1.eps}
\fi
%Figure
\end{center}
\caption{#4}
\label{fig:#2--#1}
\end{figure}
}

%%%%%MTI functions
\newcommand{\monT}[1]{\tens{M}\left(#1\right)}
\newcommand{\monTN}[2]{\tens{M}_p^{#1}\left(#2\right)}
\newcommand{\monElTN}[2]{m_p^{#1}\left(#2\right)}
\newcommand{\LieSingle}[3]{\tens{L}_{#1,#2,#3}}
\newcommand{\LieDouble}[4]{\tens{L}_{#1,#2,#3,#4}}
\def\DiffMat{\mat{\Theta}}
\def\DiffMatEl{\Theta}


%%% Definition f�r die Iterationen beim ILC
\newcommand{\xd}[1]{#1_{d}}
\newcommand{\xdi}[1]{#1_{d+1}}
\newcommand{\xdnext}[1]{#1_{d,next}}
\newcommand{\xdii}[1]{#1_{d+2}}
\newcommand{\xdb}[1]{#1_{d,db}}
\newcommand{\xdbi}[1]{#1_{d+1,db}}
\newcommand{\xdbii}[1]{#1_{d+2,db}}

